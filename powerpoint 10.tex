\documentclass{article}
\usepackage{graphics}
\usepackage{graphicx}
\begin{document}
	\section{HISTORY}
	PYTHON was derived from any other languages like C, C++ and many more scripting languages.
	During the time of implementation for PYTHON it had not yet been named but Guido van Russom got some inspiration from a comedy series called ‘ Monty Pythons Flying Circus’ and decided to name the scripting language ‘PYTHON’.
	\begin{figure}
		\includegraphics{"PYTHON"}
	\end{figure}
\paragraph{}PYTHON just like any other programming or scripting languages has different versions.
PYTHON 1.0 released on the 20th of February 1991.
PYTHON 2.0 released on the 16th of October 2000.
PYTHON 3.0 released on the 3rd of December 2008.

The language was in no way close to perfection during the early stages but later on as it progressed the flaws were fixed later on as new versions were released.
\section{APPLICATIONS}
PYTHON can be applied in the following:
\begin{itemize}
	\item GUIs
	\item SYSTEM DEVELOPMENT
	\item INTERNET SCRIPTING
\end{itemize}
\section{IDEs}
Some IDEs for PYTHON
\begin{itemize}
	\item SPYDER
	\item ATOM
	\item PyCHARM
	\item ECLIPSE
\end{itemize} 
\end{document}